%!TeX root = note
\documentclass[main.tex]{subfiles}

\begin{document}
    \chapter{Circuit Principle}

    \section{MOSFET load switch}
    \justify
    In this section, the switching operation of a MOSFET is discussed
    \begin{enumerate}
        \item MOSFET switching principle.
        \item Type of load switch.
        \item On resistance $R_{ds}$.
        \item Gate charge.
        \item Safe Operating Area.
        \item IRF4905S.
    \end{enumerate}

    \pagebreak
    \subsection{MOSFET switching principle}

    \justify
    For the N-channel MOSFET to be  "switched on" (in triode or saturation region), $V_{gs} > V_{th}$ with Gate-Source threshold voltage $V_{th} > 0$. To achieve this, a simple circuit such as in Figure \ref{fig:simple_nmos_switch} can be used. This configuration is called low-side switch where  the switch (MOSFET) is connected to the negative terminal of the supply voltage. For the P-channel MOSFET, $V_{gs} < V_{th}$ with $V_{th} < 0$ which leads to the simple circuit shown in Figure \ref{fig:simple_pmos_switch}. This configuration is called high-side switch where  the switch is connected to the positive terminal of the supply voltage.

    \begin{figure}[!h]
        \centering
        \begin{minipage}{.4\textwidth}
          \centering
          \includegraphics[width=\linewidth]{media/nmos_switch.png}
          \captionof{figure}{Low-side MOSFET switch.}
          \label{fig:simple_nmos_switch}
        \end{minipage}\qquad
        \begin{minipage}{.4\textwidth}
          \centering
          \includegraphics[width=\linewidth]{media/pmos_switch.png}
          \captionof{figure}{High-side MOSFET switch.}
          \label{fig:simple_pmos_switch}
        \end{minipage}
    \end{figure}

    \justify
    In load switching application, high-side switching is more preferable due to having no ground offset (ground loop). In a circuit exhibiting ground loop, the load is not connected directly to the ground of the power supply which can lead to distortion in signal transmission. The offset voltage is calculated by $\Delta V_\text{GND} = -I_d \cdot R_{ds}$ showing that $\Delta V_\text{GND}$ varied, and, thus, not desirable in electronics system.    

    \pagebreak
    \subsection{On resistance $R_{ds}$}
    N-channel MOSFET exhibits lower $R_{ds}$ than that of P-channel MOSFET due to packing density. It is possible to parallel two or more P-channel MOSFETs to reduce the total on resistance $R_{\sum ds}$. However, the designer has to take into account devices' parameters mismatch. An application note from Infineon and Texas Instruments \cite{InfineonParallelMOS} details the effects and solutions of each mismatches, suitable for paralleling up to 4 MOSFETs. However, \cite{MOSFET_parallel_low_power} shows that for low power application, the parameter mismatches is not as detrimental as for high power application.
    
    \pagebreak
    \subsection{Gate Charge}

    \justify
    For a MOSFET to be switched-on, the parasitic capacitors $C_{gs}$ and $C_{gd}$ modeled in Figure \ref{fig:MOSFET_parasitic_cap} has to be charged/discharged. This phenomenon is synonymous to both P-channel and N-channel MOSFET. Thus, in this section, the gate charge curve of P-channel MOSFET is discussed.

    \begin{figure}[!h]
        \centering
        \begin{minipage}{.5\textwidth}
          \centering
          \includegraphics[width=\linewidth]{media/gate_charge_curve_details.png}
          \captionof{figure}{Typical gate charge curve.}
          \label{fig:gate_charge_curve_details}
        \end{minipage}%
        \begin{minipage}{.5\textwidth}
          \centering
          \includegraphics[width=0.8\linewidth]{media/MOSFET_parasitic_cap.png}
          \captionof{figure}{MOSFET parasitic capacitance.}
          \label{fig:MOSFET_parasitic_cap}
        \end{minipage}
    \end{figure}
    
    \justify
    During a turn-on event, it can be divided into four main intervals. These intervals are shown in Figure \ref{fig:gate_charge_curve_details}.
    \begin{enumerate}
        \item $t_0 \rightarrow t_1$: $V_{gs}$ drops to approximately $V_{th}$. At this voltage, the MOSFET enters its ohmic/linear region and starts to conduct $I_d$.
        \item $t_1 \rightarrow t_2$: $V_{gs}$ drops to $V_{gs,miller}$, \textbf{Miller plateau}. At this voltage, the MOSFET is still in its linear region. The charging time and the charge is defined by:
        \begin{equation}
            \Delta t_{0\rightarrow2}=R_{g}(C_{gs} +C_{gd})ln\left(\dfrac{V_{signal-source}}{V_{gs, miller}}\right)
        \end{equation}
        \begin{equation}
            Q_{0\rightarrow2}=(C_{gs} +C_{gd})\cdot (-V_{gs, miller})
        \end{equation}
        \item $t_2 \rightarrow t_3$: While $V_{gs}$ maintains at $V_{gs,miller}$, $V_{ds}$ starts to increases to its turn-on voltage $R_{ds}(on)\cdot I_d$ of a few $mV$, and the MOSFET enters its saturation region. Because only $V_{ds}$ decreases, $C_{gd}$ is being charged with a constant current flowing through $R_{g}$. Thus, the following equation: 
        \begin{equation}
            \dfrac{\delta V_{ds}}{\delta t}=\dfrac{1}{C_{gd}}\cdot\dfrac{V_{signal-source} - V_{gs,miller}}{R_{g}} \Rightarrow \Delta V_{ds} = \dfrac{V_{signal-source} - V_{gs,miller}}{R_{g}C_{gd}}\cdot \Delta t_{2\rightarrow3}
        \end{equation}
        \begin{equation}
            Q_{2\rightarrow3}= \left| \dfrac{V_{signal-source} - V_{gs,miller}}{R_{g}} \right|\cdot \Delta t_{2\rightarrow3} 
        \end{equation}
        Ideally, in saturation region, $V_{ds} = 0V \Rightarrow \Delta V_{d} = -V_{in}$. Therefore:
        \begin{equation}
            \Delta t_{2\rightarrow3}=\dfrac{V_{signal-source}}{V_{signal-source} - V_{gs,miller}}\cdot R_{g}C_{gd}
        \end{equation}
        \begin{equation}
            Q_{2\rightarrow3}= -V_{signal-source}\cdot C_{gd}
        \end{equation}
        
        \item $t_3 \rightarrow t_4$: In the saturation region, $V_{gs}$ starts increasing to $V_{signal}$. However, $V_{gs}$ is clamped at $D1$'s zener voltage. 
    \end{enumerate}

    \justify
    In this project's application, which does not involve high speed switching, only the interval $t_0 \rightarrow t_3$ is concerned (from turn-off and saturation region). Thus, the total turn-on time $T_{on}$ and turn-on charge $Q_{on}$ is:

    \begin{equation} 
        T_{on} = R_{g}(C_{gs} +C_{gd})ln\left(\dfrac{V_{signal-source}}{V_{gs, miller}}\right) + \dfrac{V_{signal-source}}{V_{gs,miller}}R_{g}C_{gd}
    \end{equation}

    \begin{equation} \label{eq:estimated_on_charge_1}
        Q_{on} = -[(C_{gs} +C_{gd})\cdot V_{gs, miller} - V_{signal-source}\cdot C_{gd}]
    \end{equation}

    \justify
    In a MOSFET datasheet, the value of $C_{gs}$ and $C_{gd}$ are shown in terms of input capacitance $C_{iss}$ and reverse transfer capacitance $C_{rss}$ which are plotted in the datasheet such as Figure \ref{fig:typical_capacitance_plot}. The differnce between $C_{iss}$ and $C_{rss}$ are approximately constant, meaning $C_{gs}$ is constant, while $C_{gd}$ is non-linear and dependent of $V_{in}$. $V_{gs,miller}$ can be estimated by \eqref{eq:estimated_miller_voltage_TI} with $V_{th}$ and $K_n$ are unique to each MOSFET. Finally, \eqref{eq:estimated_on_charge_1} can be rewritten in \eqref{eq:estimated_on_charge_2}.

    \begin{figure}[!h]
        \centerline{\includegraphics[scale=0.25]{media/typical_capacitance_plot.png}}
        \caption{Typical capacitance vs. Drain-Source voltage.}
        \label{fig:typical_capacitance_plot}
    \end{figure}

    \begin{equation} \label{eq:estimated_miller_voltage_TI}
        V_{gs, miller} = V_{th} + \sqrt{\frac{I_{d}(\infty)}{K_n}}
    \end{equation}

    \begin{equation} \label{eq:estimated_on_charge_2}
        Q_{on} = [C_{gs} +C_{gd}(V_{in})]\cdot \left(V_{th} + \sqrt{\dfrac{I_{d}(\infty)}{K_n}} \right) + V_{signal-source}\cdot C_{gd}(V_{in})
    \end{equation}

    \justify
    For quick prototyping, the total gate charge $Q_g$ in the MOSFET's datasheet can be used. In the case of IRF4905S, $Q_g$ is $180nC$. For a charge/discharge time of, e.g.,  $t_{sw} = 100ns$, the gate current to switch the MOSFET can be approximated as $I_g = Q_g / t_{sw} = 1.8A$.

    \pagebreak
    \subsection{Safe Operating Area (SOA)}
    In a MOSFET's datasheet, the SOA of the device shows the limitation of the device's opreating range. As long as the desired application is well within the SOA, the device will, in all likelihood, not risk running into problems such as thermal runaway, material degradation, etc. A typical SOA plot of a P-channel MOSFET is shown in figure \ref{fig:typical_SOA_plot}. The SOA lies below the dashed boundary.

    \begin{figure}[!h]
        \centering
        \begin{minipage}{.5\textwidth}
          \centering
          \includegraphics[width=0.8\linewidth]{media/typical_SOA_plot.png}
          \captionof{figure}{Typical MOSFET's SOA plot.}
          \label{fig:typical_SOA_plot}
        \end{minipage}%
        \begin{minipage}{.5\textwidth}
          \centering
          \includegraphics[width=\linewidth]{media/SOA_regions.drawio.png}
          \caption{Regions within a MOSFET's SOA.}
        \label{fig:SOA_regions}
        \end{minipage}
    \end{figure}

    \justify
    The SOA can be divided into four regions shown in figure \ref{fig:SOA_regions}:
    \begin{enumerate}
        \item Region I:$-V_{ds}$ is determined purely by Ohm's law  $-V_{ds} = R_{ds}(\text{on})\cdot -I_d$ with $-I_d \leq -I_{d,\text{max}}$. Thus, region I spans from $-V_{ds} = 0$ to $-V_{ds} = \dfrac{-I_{d,\text{max}}}{R_{ds}(\text{on})}$.
        \item Region II: $-I_{d}$ is limited by $-I_{d,\text{max}}$ as a result of limitation of package, silicon and junction-to-ambient thermal impedance $R_{\theta J A}$. The power consumption $P=I_{d} * V_{ds}$ does not exceed the maximum rating. Thus, region II spans from $-V_{ds} = -I_{d,\text{max}} / R_{ds}(\text{on})$ to $-V_{ds} = \dfrac{P_\text{max}}{-I_{d,\text{max}}}$.
        \item Region III: the $-I_{d}$ is limited by $P_\text{max}$ or $-I_d = \dfrac{P_\text{max}}{-V_{ds}}$, hence, the negative coefficient observed from the plot.
        \item Region IV: The forth region is limited by thermal instability. 
    \end{enumerate}

    \justify
    Texas Instruments \cite{TISOA} noted the following:
    \begin{itemize}
        \item Region III's and IV's boundary mostly unanimous. Hence, the application should be designed to operate well within the estimated regions defined in the datasheet. 
        \item Besides region I, the other regions are soft-limited, meaning that operation outside of these regions does not guarantee immediate damage. However, the longevity of the devie is affected.
    \end{itemize}

    \justify
    To ensure the operation of the MOSFET is well within the SOA, the following basic parameters of the device should be:

    \begin{enumerate}
        \item Drain-Source breakdown voltage $V_{dss}$ is at least 20\% higher than the operating voltage. 
        \item Maximum continuous Drain current $I_{d,\text{max}}$ is at least 20\% higher than the operating current.
        \item Low on resistance $R_{ds}$ satisfying the maximum dissipation power $P_{d}=R_{ds} \cdot I_{max}$ of the MOSFET.
    \end{enumerate}

    \pagebreak
    \subsection{IRF4905S}

    \justify
    As a load switch, the following parameters are basic to the selection process:
    \begin{itemize}
        \item Safe Operating Area is guaranteed.
        \item Parsitic capacitance $C_{gd}$ does not vary largely over the operating range.
    \end{itemize}

    \justify
    Between N-channel and P-channel MOSFET, within the same price range, $V_{ds}$ and $I_d$ are similar. The main differences are the switch type (high-side or low-side) and the $R_{ds}$. These differences are discussed in details in previous sections, and is summarized in Table \ref{table:MOSFET_channel_type}
    

    \begin{table}[!h]
        \centering
        \begin{tabular}{|m{0.15\linewidth}|m{0.4\linewidth}|m{0.4\linewidth}|}
        
        \hline
        & \textbf{P-channel MOSFET} &  \textbf{N-channel MOSFET} \\
        \hline
        \textbf{Switch type} & $V_{th} < 0V \Rightarrow$ High-side switch & 
        \begin{itemize}
            \item $V_{th} > 0V \Rightarrow $ Low-side switch leads to ground offset (ground loop) varied based on $R_{ds}$ and $I_{d}$ of MOSFET. Require a gate driver IC with integrated/discrete charge pump circuit to use as high-side switch.
        \end{itemize}
        \\
        \hline
        \textbf{Typical $R_{ds}$} & 
        \begin{itemize}
            \item High, in $100 m\ohm$ range $\Rightarrow$ High power loss. By paralleling, the on resistance is halved.
            \item Low $R_{ds}$ is more expensive than its N-channel counterpart.
        \end{itemize}
         
        & Low, in $m\ohm$ range $\Rightarrow$ Low power loss.
        \\
        
        \hline
        \end{tabular}
        \caption{Comparison of P-channel \& N-channel MOSFET}
        \label{table:MOSFET_channel_type}

    \end{table}    

    \justify
    Considering the advantages and disadvantages of the two type of MOSFET transistor, it is clear that using P-channel MOSFET as high-side switch is more advantageous in terms of components counts. Finally, the IRF4905S MOSFET is used because it has a suitable margin between its $V_{ds}$ and $I_{d}$ values and a low $R_{ds}$, making it ideal for this application. 

    \begin{figure}[!h]
        \centerline{\includegraphics[scale=0.5]{media/IRF4905_key_parameters.png}}
        \caption{IRF4905 P-channel MOSFET key parameter \cite{IRF4905S}.}
        \label{fig:IRF4905S_param}
    \end{figure}    
    
    \pagebreak
    \section{MOSFET driver}
    \justify
    While the availability of N-channel MOSFET drivers for both high-side and low-side is abundance, those for P-channel MOSFET are difficult to source. Thus, in this project, a driver circuit has to be designed and built. In this section, the followings are discussed:
    \begin{enumerate}
        \item MOSFET basic switching circuit
        \item Push-pull topology.
        \item Totem pole alternative topology.
    \end{enumerate}   

    \pagebreak
    \subsection{MOSFET basic switching circuit} 
    
    An example driver circuit for P-channel MOSFET is shown in Figure \ref{fig:ideal_gate_drive}. In this circuit, the gate is charged with an ideal power supply. The Zener diode $D1$ connected across the Source-Gate of $M1$ clamps the voltage at the diode zener breakdown voltage lower than the specified maximum $V_{gs_max}$. In Figure \ref{fig:single_transistor_gate_drive}, a transistor is used to allows the small voltage control from device such as MCU, op-amp, etc.

    \begin{figure}[!h]
        \centerline{\includegraphics[width=\linewidth]{media/ideal_gate_drive.png}}
        \caption{Ideal gate driver circuit.}
        \label{fig:ideal_gate_drive}
    \end{figure}

    \begin{figure}[!h]
        \centerline{\includegraphics[width=\linewidth]{media/pmos_switch_npn.png}}
        \caption{Single-transistor gate driver circuit.} 
        \label{fig:single_transistor_gate_drive}
    \end{figure}

    \justify
    From the two example circuit, the following remarks can be made:
    \begin{enumerate}
        \item In the ideal example, the diode $D3$ allows for different current to be used for gate discharge and charge. A larger current (through both $R_2$ and $R_{gate}$) is used to charge the gate which makes the turn-off event faster than turn-on (slow turn-on). This effect is desirable for application with capacitive load to reduce in-rush current.
        \item For $Q_1$ to be in saturation region, the following equation must be satisfied:
        \begin{equation} 
        \begin{split}
            I_{B1} &> \dfrac{I_{C1}}{\beta}\\
\Longleftrightarrow \dfrac{5V - I_{E1} \cdot R_1}{R_{B1}} &> \dfrac{I_{C1}}{\beta}\\
\Longleftrightarrow 5V &> \dfrac{V_{in}}{R_{pullup} + R_1}\cdot(R_1 + \dfrac{R_{B1}}{\beta}), \quad I_{C1}\approx I_{E1} \\
\Longleftrightarrow \dfrac{5V}{V_{in}} &> \dfrac{R_1 + \dfrac{R_{B1}}{\beta}}{R_{pullup} + R_1} \\
\Longleftrightarrow \dfrac{5V}{V_{in}} \geq \dfrac{1}{6} &> \dfrac{R_1 + \dfrac{R_{B1}}{\beta}}{R_{pullup} + R_1}, \quad \textbf{max}V_{in} = 30V \\
\Longleftrightarrow R_{pullup} &> 5R_1 + \dfrac{6R_{B1}}{\beta}
        \end{split}
        \end{equation}
        \item To replicate the slow turn-on effect in the single-transistor gate driver, $I_{pullup}$ (while $Q_1$ is cut-off) needs to be larger than $I_{E1}$ (while $Q_1$ is in ohmic.
    \end{enumerate}

    \pagebreak
    \subsection{Push-pull topology}

    \justify
    In an ideal driver such as in Figure \ref{fig:gate_charge_curve_2}, the $V_{gs}$ should be as close to $0V$ for a turn-off event, and to $V_{in}$ for a turn-on event. In the case in Figure \ref{fig:NMOS_lowside_switch}, the single transistor $Q_1$ will not be driven fully into saturation or cut-off with the fixed value of $R_1$ over a large range of $V_{in}$. 

    
    \pagebreak

    \section{Overvoltage/Undervoltage Protection}

    \subsection{Overvoltage protection}

    \justify
    Reference schematic:

    \begin{figure}[!h]
        \centerline{\includegraphics[scale=0.5]{media/Zener_Crowbar_OVP.png}}
        \caption{Electrical box arrangement overview.}
        \label{fig:enclosure_arrange}
    \end{figure}

    \subsection{Undervoltage Protection}
    
    \section{Isolation Protection}

    \section{Reverse Polarity Protection}

    \pagebreak

    \section{5V Rail}

    \subsection{Principle}
    \justify
    To supply power to the protection logic circuit and the MCU, a 5V rail is needed. The following are methods of creating such constant voltage rail from the main power supply:

    \begin{table}[!h]
        \centering
        \begin{tabular}{|m{0.1\linewidth}|m{0.45\linewidth}|m{0.45\linewidth}|}
        
        \hline
        & Linear voltage regulator & Switching voltage regulator \\
        \hline
        Principle & 

        \includegraphics[width=0.45\textwidth]{media/series_regulator.png}
        The above figure illustrates a simple circuit of a linear voltage regulator. In this circuit, assuming a constant load, the voltage drop across $R_2$ is kept constant by biasing the transistor $Q$ with zener diode $DZ$. &

        \includegraphics[width=0.45\textwidth]{media/inverting_switch_regulator.png}
        The above figure illustrates a simple switching voltage regulator. In this circuit, the close-open switching of the transistor stores and releases energy of the inductor which in turn charges the capacitor maintaining a constant voltage drop $V_{out}$ across the load. \\

        \hline
        Advantages & 

        \begin{itemize}
            \item Often comes as fixed output regulator which is suitable for digital logic ICs, and microcontrollers.
            \item Requires very few external components. In the case of fixed-output, only decouply capacitors at input and ouput are needed.
        \end{itemize} &
        
        \begin{itemize}
            \item Low power loss due to on(saturation)/off switching of the transistor leading to less conduction loss, allowing for higher current application.
            \item Can perform buck ($V_{out} < V_{in}$) and boost ($V_{out} > V_{in}$).
        \end{itemize} \\

        \hline
        Disadvantages & 

        \begin{itemize}
            \item High power loss. The transistor $Q$ is in its ohmic region due to the biasing of Base-Emitter region, thus, $I_{c} \approx I_{e}$. The power loss is approximated: \newline $P_{loss} = P_{in} - P_{out} = (V_{in} - V_{load}) \cdot I_{e}$.
            \item Can only perform voltage step-down ($V_{out} < V_{in}$).
        \end{itemize} &
        
        \begin{itemize}
            \item High component count.
            \item Switching noise due to continuous charge and discharge of inductor.
            \item Lower load regulation capability in comparison to linear voltage regulation. Load regulation are heavily dependent on the LC subcircuit. 
        \end{itemize}  \\

        \hline     
        \end{tabular}
        \caption{Comparison of linear and switching voltage regulator}
        \label{table:voltage_regulator_type}
    \end{table}

    \justify
    In summary, linear voltage regulator is suitable for low power application with limited input voltage range; while switching voltage regulator is suitable for medium power consumption with limited load regulation. In this project, the MCU consumes up to $0.5A$ during wireless transmission, and the undervoltage/overvoltage logic circuit consumes negligible current.

    \begin{itemize}
        \item Linear voltage regulator: Power loss is up to $P_{loss} = (V_{in,max} - 5V) \cdot 0.5A = (30V - 5V) \cdot 0.5A = 12.5W$ which is a efficiency of $\eta = \dfrac{2.5W}{15W} = 16.67\%$
        \item Switching voltage regulator: Switching noise and voltage ripple can negatively affect MCU's data transmission.
    \end{itemize}
    
    \justify
    To offset the disadvantages of the two type of voltage regulator, a hybrid where a switching voltage regulator first regulate the input voltage to $V_{out,sw}=10V$, and a linear voltage regulator regulate $V_{out,sw}$ down to $5V$ (the desired voltage). In this setup, the power loss of the linear regulator is $P_{loss} = (V_{out,sw} - 5V) \cdot 0.5A = (10V - 5V) \cdot 0.5A = 2.5W$; switching noise and ripple is reduced by the linear regulator. $V_{out,sw}$ can be further reduced, as long as the switching regulator allows, to reduce the power loss even further.

    \justify
    Because of the invering voltage of the switching regulator, the 5V rail created is -5V in reference to the supply voltage's ground. This is also applied to $V_{out,sw}$. The undervoltage/overvoltage logic circuit will be adjusted accordingly to compensate for the negative offset.

    \subsection{MC34063AD and L7805xx}

    \justify
    MC34063AD from Onsemi \cite{MC34063AD} is chosen for the construction of the switching regulator due to its few external component requirement. The IC has a built-in BJT load switch that can withstand 1.5A and shunt current sensing for overcurrent/short protection.

    \begin{figure}[!h]
        \centerline{\includegraphics[scale=0.5]{media/MC34063AD_specs.png}}
        \caption{MC34063AD specifications.}
        \label{fig:MC34063AD_specs}
    \end{figure}

    \justify
    L7805xx series from STMicroelectronics \cite{L78} is chosen for the LDO.




    
    
\end{document}
