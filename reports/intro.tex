%!TeX root = intro
\documentclass[main.tex]{subfiles}

\begin{document}
    \section{Motivation}
    \justify
    Power monitor is critical in all fields related to electronics and electrical such as robotics, embedded system, information technology, etc. Electronic/Electrical devices require a specified operating voltage and current range to operate well, and not get damaged. Furthermore, the power consumption of a device can give insight into its performance and longevity. For example, in battery-operated system, the battery's capacity, which can be simply estimated by a voltage-capacity lookup table, should not be completely depleted to reserve its full capacity \cite{SATPATHY2021267}. Understanding the need for power monitoring, this thesis discusses the design and construction of an intermediate device connecting a power supply and a load, providing protection from overvoltage, overcurrent, over-power with configurable thresholds, and remote control and visualization.

    \pagebreak
    \section{Aims \& Objectives}
    \justify
    In this thesis, a power monitor device is designed and built whose criteria are as follows:
    \begin{itemize}
        \item Mininum operating voltage: \textbf{$10V$}.
        \item Maximum ratings for a pure resistive load:
        \begin{itemize}
            \item Input voltage: \textbf{$36V$}.
            \item Current: \textbf{$10A$}.
            \item Power: \textbf{$100W$}.
        \end{itemize}
        \item Provide \textbf{overvoltage}, \textbf{overcurrent}, \textbf{over-power} protection with configurable threshold.
        \item Connect the load to the supply voltage, and disconnect as per user's input.
        \item Upload voltage, current, power measured at the output stage to the user's device.
    \end{itemize}

    \justify
    An application running on the user's PC/laptop is needed to receive and visualize the upload data from the power monitor. The app will also provide a simple user interface to configure the protection thresholds, and send a signal to disconnect the load from the power supply.

    \pagebreak
    \section{Outline of Thesis}
    \justify
    The thesis consists of 7 chapters:

    \begin{enumerate}[label=\textbf{Chapter \arabic*}, leftmargin=*]
        \item Introduction: \newline Motivation, goals and objectives of the thesis.
        \item Circuit design: \newline The design process of each component of the device's circuit, and how each component connects with each other is discussed.
        \item PCB layout: \newline The key design points of the PCB layout are discussed.
        \item Firmware: \newline The microcontroller unit (MCU)'s firmware is broken down into tasks which are discussed in details.
        \item Application: \newline The application running on the client's side design pattern and operation are discussed.
        \item Result: \newline Any issues of both the power monitor device and the application are discussed, along with any potential improvements.
        \item Conclusion.
    \end{enumerate}

\end{document}